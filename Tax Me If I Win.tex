%% LyX 2.3.2 created this file.  For more info, see http://www.lyx.org/.
%% Do not edit unless you really know what you are doing.
\documentclass[english]{article}
\usepackage[T1]{fontenc}
\usepackage[latin9]{inputenc}
\usepackage{babel}
\usepackage{textcomp}
\usepackage{amstext}
\usepackage[authoryear]{natbib}
\usepackage[unicode=true]
 {hyperref}

\makeatletter

%%%%%%%%%%%%%%%%%%%%%%%%%%%%%% LyX specific LaTeX commands.
%% Because html converters don't know tabularnewline
\providecommand{\tabularnewline}{\\}

\makeatother

\begin{document}

\subsection{Survey ``Beliefs climate policies''\label{subsec:Survey-Beliefs-climate}}

\subsubsection{Survey Data Collection }

The 3002 responses were collected in February and March 2019 through
the survey company Bilendi. This company maintains a panel of French
respondents whom they can email with survey links. Respondents are
paid 2.50� if they fully complete the survey. The respondents who
choose to respond are first channeled through some screening questions
that ensure that the final sample is representative along six socio-demographic
characteristics: gender, age (5 brackets), education (4), socio-professional
category (8), size of town (5) and region (9). The quotas are relaxed
by 5\% to 10\% relative to actual proportions. Table 1 confirms that
our sample is very representative. Nonetheless, observations are thus
weighted to correct small differences between sample and population
proportions. The median time for completion of the survey was 19 minutes.

\begin{table}
\caption{\label{tab:Sample-Characteristics}Sample Characteristics}
\begin{tabular}{ccc}
\hline 
\textbf{gender} & woman & man\tabularnewline
\hline 
\emph{Population} & \emph{0.52} & \emph{0.48}\tabularnewline
Sample & 0.53 & 0.47\tabularnewline
\hline 
\end{tabular}\qquad{}%
\begin{tabular}{cccccc}
\hline 
\textbf{age} & 18-24 & 25-34 & 35-49 & 50-64 & >65\tabularnewline
\hline 
\emph{Population} & \emph{0.12} & \emph{0.15} & \emph{0.24} & \emph{0.24} & \emph{0.25}\tabularnewline
Sample & 0.11 & 0.11 & 0.24 & 0.26 & 0.27\tabularnewline
\hline 
\end{tabular}\bigskip{}
\begin{tabular}{ccccccccc}
\hline 
\textbf{profession} & farmer & independent & executive & intermediate & employee & worker & retired & inactive\tabularnewline
\hline 
\emph{Population} & \emph{0.01} & \emph{0.03} & \emph{0.09} & \emph{0.14} & \emph{0.15} & \emph{0.12} & \emph{0.33} & \emph{0.12}\tabularnewline
Sample & 0.01 & 0.04 & 0.09 & 0.14 & 0.16 & 0.13 & 0.33 & 0.11\tabularnewline
\hline 
\end{tabular}\bigskip{}
\begin{tabular}{cccccc}
\hline 
\textbf{size of town} & rural & -20.00 & 20-99 & >100 & Paris area\tabularnewline
\hline 
\emph{Population} & \emph{0.22} & \emph{0.17} & \emph{0.14} & \emph{0.31} & \emph{0.16}\tabularnewline
Sample & 0.24 & 0.18 & 0.13 & 0.29 & 0.15\tabularnewline
\hline 
\end{tabular}\bigskip{}
\begin{tabular}{ccccc}
\hline 
\textbf{education} & No diploma or \emph{Brevet} & \emph{CAP} or \emph{BEP} & \emph{Bac} & Higher\tabularnewline
\hline 
\emph{Population} & \emph{0.30} & \emph{0.25} & \emph{0.17} & \emph{0.29}\tabularnewline
Sample & 0.24 & 0.26 & 0.18 & 0.31\tabularnewline
\hline 
\end{tabular}\bigskip{}
\begin{tabular}{cccccccccc}
\hline 
\textbf{region} & \emph{IDF} & \emph{Nord} & \emph{Est} & \emph{SO} & \emph{Centre} & \emph{Ouest} & \emph{Occ} & \emph{ARA} & \emph{PACA}\tabularnewline
\hline 
\emph{Population} & \emph{0.19} & \emph{0.09} & \emph{0.13} & \emph{0.09} & \emph{0.10} & \emph{0.10} & \emph{0.09} & \emph{0.12} & \emph{0.08}\tabularnewline
Sample & 0.17 & 0.10 & 0.12 & 0.09 & 0.12 & 0.10 & 0.08 & 0.13 & 0.08\tabularnewline
\hline 
\end{tabular}

\end{table}


\subsubsection{The Survey }

The full survey in French can be seen \href{https://hbs.qualtrics.com/jfe/form/SV_9zqdJWZXgpWfjsF}{on-line}.\footnote{\href{http://hbs.qualtrics.com/jfe/form/SV_9zqdJWZXgpWfjsF}{hbs.qualtrics.com/jfe/form/SV\_9zqdJWZXgpWfjsF}}
It contains several random branches and treatments that are independent
from one another.

\paragraph{Priming On Environmental Issues}

Two blocks of information are randomly displayed or not: one on climate
change and the other of particulate matter. This priming divides the
sample in four groups, who received either one block of information,
the other, none or both of them. Climate change information includes
temperature trends for long run future, worrying facts on current
and expected impacts, and a claim that keeping global warming below
2�C is technically feasible. Particulate information consists in the
estimated impact on French mortality (48,000 deaths per year), life
expectancy (9 months less), and the assertion that reducing fuel consumption
would improve health. The time spent on each block is saved, and links
to scientific references are provided to support the information.

\paragraph{Household Characteristics}

In addition to the quotas strata, socio-demographic characteristics
include zip code, household structure, income of the respondent and
of their household. Energetic characteristics contain questions that
allow us to estimate the impact of a carbon tax increase on housing
expenditures: surface of accommodation, heating type (collective or
individual) and energy source; as well as on transport expenditures:
number of vehicles, type(s) of fuel, distance traveled last year and
average fuel economy. The distributions of answers are much in-line
with official statistics. TODO

\paragraph{Partial Tax Reforms}

To study the perception of different indirect taxes, respondents are
first asked using a Likert scale whether they would loose more or
less purchasing power than the average French household through an
increase of the VAT tax (where both additional rate and usage of the
revenues remain unspecified). Then, one partial tax reform is randomly
allocated to the respondent: it consists in an increase of the carbon
tax by 50�/t$\text{CO}_{2}$ specific either to heating fuel and gas,
or to gasoline and diesel. Partial reforms on housing and transport
feature the same questions. Similarly to the VAT question, it starts
with the relative loss of purchasing power. Then, we specify in a
new block that the revenues of the tax would be distributed equally
to each adult, entailing a yearly transfer of 50� (resp. 60�) for
the partial tax on housing (resp. transport). We also provide the
price increases implied by the tax: +13\% (resp. +15\%) for gas (resp.
domestic fuel) on the one hand, and +0.11� (resp. +0.13�) for a liter
of gasoline (resp. diesel) on the other. Then, we ask the respondent
whether their household would win, loose or be unaffected by the reform,
and further ask them to estimate their subjective gain or loss among
5 or 6 intervals. The interval thresholds are tailored to each respondent,
as they are computed in proportion of the number of consumption units
of their household (as defined by \href{http://ec.europa.eu/eurostat/statistics-explained/index.php/Glossary:Equivalised_disposable_income}{Eurostat}).
Finally, respondents are asked to estimate their own elasticity as
well as that of French people. To this end, we borrow the phrasing
of \citet{baranzini_effectiveness_2017}, and ask the expected decrease
in consumption that would follow a 30\% increase in the price of heating
(or equivalently, an increase of 0.50� in fuel prices), among 5 brackets. 

\paragraph{Compensated Carbon Tax Increase}

\subparagraph{Perceptions \emph{ex ante}}

Our main reform of interest is an increase by 50�/t$\text{CO}_{2}$
of the French carbon tax, that concerns both housing and transport.
The revenues generated are redistributed equally, so that each adult
<<<<<<< refs/remotes/thomasdouenne/master
<<<<<<< refs/remotes/thomasdouenne/master
receives a yearly compensation of 110�. This amount was computed using
=======
receives a yearly compensation of 110�. This amount was computed using
>>>>>>> create papier.R to put all our result in a linear order
=======
receives a yearly compensation of 110�. This amount was computed using
>>>>>>> Auto stash before rebase of "origin/master"
the estimations of \citet{douenne_vertical_2019} for the elasticity:
$-0.4$ for transport and $-0.2$ for housing, as well as an incidence
borne at 80\% by the consumers. After describing the reform, a first
block of questions elicits the respondent's perceptions. Their subjective
gain is asked in the same manner than for the partial tax. The priming
that ``scientists agree that a carbon tax would be effective in reducing
pollution'' is randomly displayed or not before asking whether the
reform would be effective in reducing pollution and fighting climate
change. Then, respondents are asked to pick the categories of losers
and winners. Finally, we ask: ``Would you approve this reform?''
and let the respondent choose between ``Yes'', ``No'' or ``PNR
(I don't know, I don't want to answer)''. In the following, we say
that a respondent \emph{approve }a reform if they respond ``Yes'',
and that they \emph{accept} the reform if they do \emph{not} respond
``No''. Given the low rates of approval, we study primarily the
acceptance to get tighter confidence intervals.

\subparagraph{Opinion \emph{after knowledge}}

To test the persistence of beliefs and measure the importance of self-interest
and fairness motives in the acceptance of the reform, we provide some
information on the effect of the reform. To a random half of the sample,
we explain that ``this reform would increase the purchasing power
of the poorest households and decrease that of the richest, who consume
more energy''. To two-thirds of the respondents (the remaining half
plus one third of the respondent with the priming on \emph{progressivity}),
we tell that: ``In five cases over six, a household with your characteristics
would {[}win/loose{]} through the reform. (The characteristics taken
into account are: heating using {[}energy source{]} for an accommodation
of {[}surface{]} m$^{2}$; {[}distance{]} km traveled with an average
consumption of {[}fuel economy{]} L for 100 km.)''. Indeed, section
\ref{subsec:Consumer-survey-Budget} shows that we estimate correctly
if a household wins or loose in 83\% of cases. 

During the survey collection, we understood that most respondents
did not believe the claim that the reform was progressive, so we asked
to the second half of the sample whether they thought it was, to analyze
the effects of priming on these compliant respondents. We also ask
again the winning category (i.e. if the respondent's household would
win, loose or be unaffected by the reform) and the approval to the
reform. Finally, we let the respondent pick the reasons why this reform
seems beneficial, and undesirable. TODO: stats des

\paragraph{Tax Increase with Targeted Compensation}

In this block, we ask for the winning category and for the approval
of four alternative reforms. Each respondent deals with just one of
them, which differ from the main reform only in the way revenues are
recycled. Here, the payments, still equal among recipients, are targeted
to adults whose income is below some threshold. The four threshold
correspond to the bottom 20\%, 30\%, 40\% and 50\% of the income distribution.
They are computed using inflated deciles of individual income from
the \emph{Enqu�te sur les Revenus Socio-Fiscaux }(ERFS 2014) produced
by \emph{INSEE }(the French national statistics bureau).\footnote{Incomes entitled to the household rather to its members, such as certain
welfare benefits, are divided equally among the two oldest adults
of the household.} Respondents whose income lies between two thresholds are allocated
randomly to a reform featuring one of them. When the income is close
to only one threshold (i.e. when its percentile in the distribution
is below 20 or within $\left[50;70\right]$), the reform allocated
corresponds to that one. When the respondent's income is above 2220�/month
(which is the 70th. percentile), the reform they face is determined
by the income of their spouse. Finally, when both (or the only one)
adults in the household earn more than 2220�/month, their reform is
allocated randomly between the four variants. Table \ref{tab:Compensation-amount-by}
describes the targeted reforms and the proportion of respondents allocated
to each of them, along with the proportion one would expect from the
\emph{ERFS}. The two sets of figures matches almost perfectly, indicating
that our sample is representative along the income dimension.

\begin{table}
\caption{\label{tab:Compensation-amount-by}Characteristic of the targeted
reform by target of the payment.}
\centering%
\begin{tabular}{ccccc}
\hline 
Targeted percentiles & $\leq20$ & $\leq30$ & $\leq40$ & $\leq50$\tabularnewline
\hline 
Income threshold (�/month) & 780 & 1140 & 1430 & 1670\tabularnewline
Payment to recipients (�/year) & 550 & 360 & 270 & 220\tabularnewline
Proportion of respondents & .356 & .152 & .163 & .329\tabularnewline
\emph{Expected proportion of respondents} & \emph{.349} & \emph{.156} & \emph{.156} & \emph{.339}\tabularnewline
\hline 
\end{tabular}

\end{table}


\paragraph{Other Questions}

We do not detail the other questions of the survey, because we analyze
them in a companion paper. TODO:ref We scrutinize opinions on environmental
policies, including other ways to recycle the revenues of a carbon
tax. We measure the knowledge and perceptions of climate change; ask
some specific questions over shale gas, over the influence of climate
change on the choice to give birth and on willingness to change one's
lifestyle. We study the use, availability and satisfaction with public
transportation and active mobility. We ask political preferences,
including the positioning in relation to \emph{yellow vests}. Finally,
we let the respondent express any comment in a text box.

\subsubsection{Ensuring Data Quality }

We took several steps to ensure the best possible data quality. We
excluded the 2\% of respondents who spend less than 7 minutes on the
full survey. We confirm that response time is not significantly correlated
with our variable of interest (such as approval or subjective gain).
In order to screen out inattentive respondents, a test of quality
of the responses was inserted, which asked to select ``A little''
on a Likert scale. The 9\% of respondents who failed the test were
also excluded from our final sample of 3002 respondents. Also, when
the questions about a reform were spread over different pages, we
recalled the details of the reform on each new page. We check for
careless or strange answers on numerical questions, such as income
or the size of the household. We flag 10 respondents with aberrant
answers to the size of the household (above 12) and up to 246 respondents
with incorrect answers, such as a size of household smaller than the
number of adults, or a household income smaller than individual income.
An examination of these answers shows no significant correlation with
our variables of interest, and suggests that these respondents have
simply mistaken the question (e.g. they may have confounded the zip
code with the household size, or household income with individual
income). Also, 58 respondents have answered more than 10,000� as their
monthly income (despite the word ``monthly'' being in bold and underlined),
with answers in the typical range of French incomes. We have divided
these figures by 12.

\subsection{Consumer survey ``Budget de Famille''\label{subsec:Consumer-survey-Budget}}

<<<<<<< refs/remotes/thomasdouenne/master
<<<<<<< refs/remotes/thomasdouenne/master
=======
>>>>>>> Auto stash before rebase of "origin/master"
\section{Review of the Literature}

\citet{stern_value_1993} is an early work proposing and testing a
model of preferences for environmental quality aimed at disentangling
egoistic from altruistic motives on the one hand, and beliefs from
values on the other hand. Among all possible attitudes, they show
that belief about consequences on self-interest is the only predictor
of the willingness to pay Pigouvian taxes. Using a post-electoral
survey in Switzerland, \citet{thalmann_public_2004} studies socio-demographic
determinants for the approval of environmental policies, and finds
a positive correlative with having a secondary degree or an affinity
for left-wing or green parties. Similarly to \citet{stern_value_1993},
he does not find a significant correlation for age or gender, but
he finds one for self-interest, proxied by the number of cars owned.
Surveying Norwegian people, \citet{kallbekken_you_2011} shows that
self-interest matters for acceptance, but less than concerns for environmental
effectiveness or distributional effects. In the present paper, we
also study how these three motives affect acceptance. We contribute
to the literature by providing evidence for causal effects where \citet{kallbekken_public_2011}
essentially show correlations. Besides, \citet{kallbekken_public_2011}
study the three motives through proxies, such as fuel consumption
used for self-interest. On the contrary, we do not assume that people
are rational nor have perfect information, and our methodology allows
to disentangle erroneous beliefs from \emph{pure }effects of self-interest
or progressivity. In surveys on British and Swedish respondents respectively,
\citet{bristow_public_2010} and \citet{brannlund_tax_2012} document
a higher approval when the reform addresses distributional issues.
\citet{bristow_public_2010} find that respondents focus on costs
and do not discuss gain even when they are made salient, either because
of loss aversion or distrust than the government would distribute
a lump sum transfer. Their finding that people prefer that a non-for
profit organize the scheme rather than the government is consistent
with that of \citet{alesina_intergenerational_2018} that only 17\%
of British and 6\% of French people trust the government. This generalized
distrust resonates with the more specific suspicion against the environmental
objective of the reform we find, as 62\% of our respondents think
that either ``tax and dividend'' is a pretext to increase taxes
or that it is not effective to reduce pollution. Similarly, \citet{baranzini_effectiveness_2017}
report that a majority of the people they interviewed in Geneva do
not believe the tax would be effective, which confirms what \citet{dresner_history_2006}
find with focus groups in the UK.

Furthermore, several papers study how approval of a carbon tax varies
with usage of its revenues \citep{thalmann_public_2004,kallbekken_you_2011,saelen_choice_2011,gevrek_public_2015,baranzini_effectiveness_2017}.
A common finding is that earmarking the revenues to environmental
policies increases approval, which can be explained by a stronger
belief in environmental effectiveness \citep{saelen_choice_2011}.
We document a similar pattern in our (more descriptive) companion
paper. 

Some papers show lower support when the price instrument is labeled
as a ``tax'' \citep{kallbekken_you_2011,brannlund_tax_2012,baranzini_effectiveness_2017}.
The French government probably took this result into account when
creating the ``Climate-Energy Contribution''. However, this does
not fool anyone anymore. Nowadays, the media, the yellow vests and
the government all plainly call this instrument a \emph{tax}. As it
is unlikely that the government can still use labeling as a nudge
in this context, we do not study this ---already well-documented---
effect.

For a more complete review of the literature, we refer the reader
to the thorough survey of \citet{carattini_overcoming_2018}. We also
signal the less recent \citet{dresner_social_2006} and the more synthetic
\citet{klenert_making_2018}.
<<<<<<< refs/remotes/thomasdouenne/master
=======
\citealp{edwards_conservatism_1968}\citep{benabou_self-confidence_2002,eil_good_2011,coutts_good_2018}
>>>>>>> create papier.R to put all our result in a linear order
=======
>>>>>>> Auto stash before rebase of "origin/master"

\bibliographystyle{\string"C:/Users/a.fabre/Google Drive/Economie/Articles/ANDplainnat\string"}
\bibliography{/var/www/bibs/CO2_tax_acceptability,\string"C:/Users/a.fabre/Google Drive/Economie/Articles/CO2_tax_acceptability\string"}

\end{document}
