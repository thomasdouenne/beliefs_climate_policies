%% LyX 2.3.0 created this file.  For more info, see http://www.lyx.org/.
%% Do not edit unless you really know what you are doing.
\documentclass[english]{article}
\usepackage[T1]{fontenc}
\usepackage[latin9]{inputenc}
\usepackage{textcomp}
\PassOptionsToPackage{normalem}{ulem}
\usepackage{ulem}
\usepackage{babel}
\begin{document}
Hereafter, we only describe questions of the survey that are used
in the present paper. The other questions are described and analyzed
in our companion paper, TODO:cite.

\paragraph{Socio-demographics}
\begin{enumerate}
\item What is your postal code? 
\item What is your gender (in the sense of civil status)? \emph{}\\
\emph{Female; Male }
\item What is your age group? \emph{}\\
\emph{18 to 24 years old; 25 to 34 years old; 35 to 49 years old;
50 to 64 years old; 65 years old or more} 
\item What is your employment status? \emph{}\\
\emph{Permanent; Temporary contract; Unemployed; Student; Retired;
Other active; Inactive}
\item What is your socio-professional category? (Remember that the unemployed
are active workers). \emph{}\\
\emph{Farmer; Craftsperson, merchant; Independent; Executive; Intermediate
occupation; Employee; Worker; Retired; Other Inactive} 
\item What is your highest degree? \emph{}\\
\emph{No diploma; Brevet des coll�ges; CAP or BEP {[}secondary{]};
Baccalaureate; Bac +2 (BTS, DUT, DEUG, schools of health and social
training...); Bac +3 (licence...) {[}bachelor{]}; Bac +5 or more (master,
engineering or business school, doctorate, medicine, master, DEA,
DESS...)}
\item How many people live in your household? Household includes: you, your
family members who live with you, and your dependents. 
\item What is your net \textbf{\uline{monthly}} income (in euros)? \textbf{\uline{All
income}} (before withholding tax) is included here: salaries, pensions,
allowances, APL {[}housing allowance{]}, land income, etc. 
\item What is the net \textbf{\uline{monthly}} income (in euros) \textbf{\uline{of
your household}}? \textbf{\uline{All income}} (before withholding
tax) is included here: salaries, pensions, allowances, APL {[}housing
allowance{]}, land income, etc. 
\item In your household how many people are 14 years old or older (\textbf{\uline{including
yourself}})? 
\item In your household, how many people are over the age of majority (\textbf{\uline{including
yourself}})? 
\end{enumerate}

\paragraph{Energy characteristics}
\begin{enumerate}
\item What is the surface area of your home? (in m\texttwosuperior )
\item What is the heating system in your home? \emph{}\\
\emph{Individual heating; Collective heating; PNR (Don't know, don't
say)}
\item What is the main heating energy source in your home? \emph{}\\
\emph{Electricity Town gas; Butane, propane, tank gas; Heating oil;
Wood, solar, geothermal, aerothermal (heat pump); Other; PNR (Don't
know, don't say)}
\item How many motor vehicles does your household have? \emph{}\\
\emph{None; One; Two or more} 
\item {[}Without a vehicle{]} How many kilometers have you driven in the
last 12 months? 
\item {[}One vehicle{]} What type of fuel do you use for this vehicle? \emph{}\\
\emph{Electric or hybrid; Diesel; Gasoline; Other} 
\item {[}One vehicle{]} What is the average fuel economy of your vehicle?
(in Liters per 100 km)
\item {[}One vehicle{]} How many kilometers have you driven with your vehicle
in the last 12 months?
\item {[}At least two vehicles{]} What type of fuel do you use for your
main vehicle?\\
 \emph{Electric or hybrid; Diesel; Gasoline; Other} 
\item {[}At least two vehicles{]} What type of fuel do you use for your
second vehicle?\\
 \emph{Electric or hybrid; Diesel; Gasoline; Other} 
\item {[}At least two vehicles{]} What is the average fuel economy of all
your vehicles? (in Liters per 100 km) 
\item {[}At least two vehicles{]} How many kilometers have you driven with
all your vehicles in the last 12 months? 
\end{enumerate}

\paragraph{Partial reforms {[}transport / housing{]}}

(...)\emph{}
\begin{enumerate}
\item If fuel prices increased by 50 cents per liter, by how much would
\textbf{\uline{your household}} reduce its fuel consumption? \emph{}\\
\emph{0\% -} {[}\emph{I already consume almost none }/\emph{ I am
already not consuming}{]}\emph{; 0\% - }{[}\emph{I am constrained
on all my trips} / \emph{I will not reduce it}{]}\emph{; From 0\%
to 10\%; From 10\% to 20\%; From 20\% to 30\%; More than 30\% - }{[}\emph{I
would change my travel habits significantly }/ \emph{I would change
my consumption significantly}{]}
\item In your opinion, if {[}fuel prices increased by 50 cents per liter
/ gas and heating oil prices increased by 30\%{]}, by how much would
\textbf{\uline{French people}} reduce their consumption on average?
\emph{}\\
\emph{From 0\% to 3\%; From 3\% to 10\%; From 3\% to 10\%; From 10\%
to 20\%; From 20\% to 30\%; More than 30\%} 
\end{enumerate}

\paragraph{Tax \& dividend: initial}
\begin{enumerate}
\item The government is studying an increase in the carbon tax, whose revenues
would be redistributed to all households, regardless of their income.
This would imply: 
\end{enumerate}
\begin{itemize}
\item an increase in the price of gasoline by 11 cents per liter and diesel
by 13 cents per liter; 
\item an increase of 13\% in the price of gas, and 15\% in the price of
heating oil;
\item an annual payment of 110� to each adult, or 220� per year for a couple.
\\
\\
(...)
\end{itemize}
\begin{enumerate}
\item {[} {[}empty{]} / Scientists agree that a carbon tax would be effective
in reducing pollution.{]} Do you think that such a measure would reduce
pollution and fight climate change? \emph{}\\
\emph{Yes; No; PNR (Don't know, don't say)}
\item In your opinion, which categories would lose {[} {[}blank{]} / purchasing
power{]} with such a measure? (Several answers possible) \emph{}\\
\emph{No one; The poorest; The middle classes; The richest; All French
people; Rural or peri-urban people; Some French people, but not a
particular income category; PNR (Don't know, don't say)} 
\item In your opinion, what categories would gain purchasing power with
such a measure? (Several answers possible) \emph{}\\
\emph{No one; The poorest; The middle classes; The richest; All French
people; Urban dwellers; Some French people, but not a particular income
category; PNR (Don't know, don't say)} 
\end{enumerate}

\paragraph{Tax \& dividend: after knowledge}

We always consider the same measure. 
\begin{enumerate}
\item Why do you think this measure is beneficial? (Maximum three responses)
\emph{}\\
\emph{Contributes to the fight climate change; Reduces the harmful
effects of pollution on health; Reduces traffic congestion; Increases
my purchasing power; Increases the purchasing power of the poorest;
Fosters France's independence from fossil energy imports; Prepares
the economy for tomorrow's challenges; For none of these reasons;
Other (specify): }
\item Why do you think this measure is unwanted? (Maximum three answers)
\emph{}\\
\emph{Is ineffective in reducing pollution; Alternatives are insufficient
or too expensive; Penalizes rural areas; Decreases my purchasing power;
Decreases the purchasing power of some modest households; Harms the
economy and employment; Is a pretext for raising taxes; For none of
these reasons; Other (specify):} 
\end{enumerate}

\paragraph{Attitudes over other policies}
\begin{enumerate}
\item In which cases would you be in favor of increasing the carbon tax?
I would be in favor if the tax revenues were used to finance...\emph{ }
\begin{enumerate}
\item a payment to the 50\% poorest French people (those earning less than
1670� per month) 
\item a payment to all French people 
\item a compensation for households forced to consume petroleum products
\item a decrease in social contributions
\item a decrease in VAT 
\item a decrease in the public deficit 
\item the thermal renovation of buildings 
\item renewable energy (wind, solar, etc.) 
\item clean transport
\end{enumerate}
\end{enumerate}
\emph{Yes, absolutely; Yes, rather; Indifferent or Don't know; No,
not really; No, not at all}
\begin{enumerate}
\item Please select ``A little'' (test to check that you are attentive).
\emph{}\\
\emph{Not at all; A little; A lot; Completely; PNR (Don't know, don't
say)} 
\item Would you support the following environmental policies? 
\begin{enumerate}
\item A tax on kerosene (aviation) 
\item A tax on red meat 
\item Stricter standards on the insulation of new buildings 
\item Stricter standards on the pollution of new vehicles Stricter standards
on pollution during roadworthiness tests 
\item The prohibition of polluting vehicles in city centers 
\item The introduction of urban tolls 
\item A contribution to a global climate fund 
\end{enumerate}
\end{enumerate}
\emph{Yes, absolutely; Yes, rather; Indifferent or Don't know; No,
not really; No, not at all}
\begin{enumerate}
\item For historical reasons, diesel is taxed less than gasoline. Would
you be in favor of raising taxes on diesel to catch up with the level
of taxation on gasoline? \emph{}\\
\emph{Yes; No; PNR (Don't know, don't say) }
\end{enumerate}

\paragraph{Attitudes over climate change}
\begin{enumerate}
\item How often do you talk about climate change? \emph{}\\
\emph{Several times a month; Several times a year; Almost never; PNR
(Don't know, don't say) }
\item In your opinion, climate change... \emph{}\\
\emph{is not a reality; is mainly due to natural climate variability;
is mainly due to human activity; PNR (Don't know, don't say). }
\item Which of the following elements contribute to global warming? (Several
answers possible) \emph{}\\
\emph{CO$_{2}$; Methane; Oxygen; Particulate matter}
\item In your opinion, which of the following statements are true? (Several
answers possible). \emph{}\\
\emph{Consuming one beef steak emits about 20 times more greenhouse
gases than eating two servings of pasta.; Electricity produced by
nuclear power emits about 20 times more greenhouse gases than electricity
produced by wind turbines.; A seat in a Bordeaux - Nice journey emits
about 20 times more greenhouse gases by plane than by high speed train. }
\item In your opinion, how would the effects of climate change be, if humanity
did nothing to limit it? \emph{}\\
\emph{Insignificant, or even beneficial; Small, because humans would
be able to live with it; Grave, because there would be more natural
disasters; Disastrous, lifestyles would be largely altered; Cataclysmic,
humankind would disappear; PNR(Don't know, don't say) }
\item In which of these two regions do you think will climate change have
the worst consequences? \emph{}\\
\emph{The European Union; India; As much in both }
\item In your opinion, in France, which generations will be seriously affected
by climate change? (Several answers possible) \emph{}\\
\emph{People born in the 1960s; People born in the 1990s; People born
in the 2020s; People born in the 2050s; None of the four }
\item In your opinion, who is responsible for climate change? (Several possible
choices) \emph{}\\
\emph{Each of us; The richest; Governments; Some foreign countries;
Past generations; Natural causes }
\item Currently, each French person emits on average the equivalent of 10
tons of CO$_{2}$ per year. \\
\\
In your opinion, how much must this figure be reduced to by 2050 in
order to hope to contain global warming to +2�C in 2100 (if all countries
did the same)? In 2050, we should emit at most... \emph{}\\
\emph{0; 1; 2; 3; 4; 5; 6; 7; 8; 9; 10} tons 
\item Has climate change had or will it have an influence on your decision
to make a child (or children)?\emph{ }\\
\emph{Yes; No; PNR (Don't know, don't say)}
\item {[}If \emph{Yes}{]} Why does climate change influence your decision
to have a child (or children)? (Several answers possible). \emph{}\\
\emph{Because I don't want my child to live in a devastated world.;
Because each additional human being aggravates climate change.}
\item Would you be willing to change your lifestyle to fight climate change?
(Several answers possible) \emph{}\\
\emph{Yes, if policies went in this direction; Yes, if I had the financial
means; Yes, if everyone did the same; No, only the richest people
have to change their way of life; No, it is against my personal interest;
No, I think climate change is not a real problem; I have already adopted
a sustainable way of life; I try, but I have trouble changing my habits} 
\item Assuming that all states in the world agree to firmly fight climate
change, notably through a transition to renewable energy, through
efforts of the richest, and imagining that France would expand the
supply of non-polluting transport very widely; would you be willing
to adopt an ecological lifestyle (i.e. eat little red meat and ensure
to use almost no gasoline, diesel or kerosene)? \emph{}\\
\emph{Yes; No; PNR (Don't know, don't say) }
\end{enumerate}

\paragraph{Shale gas (and smoking)}
\begin{enumerate}
\item Do you smoke regularly? \emph{Yes; No }
\item The use of shale gas would limit climate change, as gas would be exported
and used to produce electricity instead of coal. On the other hand,
extraction would risk reducing water quality at the local level. Your
department {[}would be / would not be{]} concerned by the exploitation
of shale gas. \\
\\
In view of this information, would you be in favor of shale gas exploitation
in France? \emph{}\\
\emph{Yes; No; PNR (Don't know, don't say) }
\item What would be the main benefit to you from shale gas development?
\emph{}\\
\emph{This would limit climate change; This would create jobs and
boost the department; None of these two reasons }
\item What do you think of the idea that shale gas would limit climate change?
\emph{}\\
\emph{It is valid: any decrease in emissions goes in the right direction;
It is unwelcome: emissions should be stopped, not just slowed down;
PNR(Don't know, don't say) }
\end{enumerate}

\paragraph{Access to public transport and mobility habits}
\begin{enumerate}
\item How many minutes walk is it to the nearest public transit stop? (To
simplify, you can use the conversion 1 km = 10 min walk). \emph{}\\
\emph{in min:} ; \emph{PNR (Don't know, don't say) }
\item How often does the nearest public transport pass? (excluding school
buses) \emph{}\\
\emph{Less than three times a day; Between four times a day and once
an hour; Once or twice an hour; More than three times an hour; PNR
(Don't know, don't say) }
\item What do you think about the availability of public transport where
you live? It is... \emph{}\\
\emph{Satisfactory; Suitable, but should be increased; Limited, but
sufficient; Insufficient; PNR (Don't know, don't say) }
\item What mode of transportation do you mainly use for each of the following
trips?
\begin{enumerate}
\item Home - work (or studies) 
\item Grocery shopping 
\item Leisure (excluding holidays) 
\end{enumerate}
\end{enumerate}
\emph{Car; Public transport; Walking or cycling; Two-wheeled vehicle;
Carpooling;} \emph{Not concerned} 
\begin{enumerate}
\item {[}If \emph{Car }selected for Work{]} Would it be possible for you,
without changing your home or workplace, to travel from home to work
using public transport? \emph{}\\
\emph{Yes, it would not be very difficult for me; Yes, but it would
bother me; No; PNR (Don't know, don't say) }
\item {[}If \emph{Car }selected for Work{]} Would it be possible for you,
without changing your home or workplace, to travel from home to work
by walking or cycling? \emph{}\\
\emph{Yes, it would not be very difficult for me; Yes, but it would
bother me; No; PNR (Don't know, don't say) }
\end{enumerate}

\paragraph{Politics and media}
\begin{enumerate}
\item How much are you interested in politics? \emph{}\\
\emph{Almost not; A little; A lot }
\item How would you define yourself? (Several answers possible) \emph{}\\
\emph{Extreme left; Left; Center; Right; Extreme right; Liberal; Conservative;
Liberal; Humanist; Patriot; Apolitical; Ecologist }
\item How do you keep yourself informed of current events? Mainly through...
\emph{}\\
\emph{Television; Press (written or online); Social networks; Radio;
Other}
\item What do you think of the Yellow Vests? (Several answers possible)
\emph{}\\
\emph{I am part of them; I support them; I understand them; I oppose
them; PNR (Don't know, don't say) }
\end{enumerate}

\paragraph{Open field}
\begin{enumerate}
\item The survey is nearing completion. You can now enter any comments,
comments or suggestions in the field below.
\end{enumerate}

\end{document}
